\documentclass[compress]{beamer}

%\setbeamercovered{dynamic}

\usepackage[nofonts]{ctex}
\setCJKmainfont[ItalicFont={Kaiti SC}]{Kaiti SC}%
%\setCJKmainfont[ItalicFont={AR PL KaitiM GB}]{AR PL KaitiM GB}%
%\setCJKsansfont{WenQuanYi Zen Hei}% 文泉驿的黑体

\mode<beamer>
{
    \useinnertheme{rounded}
    \useoutertheme{split}
    \usecolortheme{rose}
    \usecolortheme{seahorse}

    \expandafter\def\expandafter\insertshorttitle\expandafter{%
    \insertshorttitle\hfill%
    \insertframenumber\,/\,\inserttotalframenumber}
}

\mode<handout>
{
	\usetheme{default}
	\usepackage{pgfpages}
	\pgfpagesuselayout{4 on 1}[a4paper,landscape,border shrink=5mm]
}


\usepackage{amsmath,latexsym,amssymb,amsfonts,amsbsy}
\usepackage{graphicx}
\usepackage{hyperref}
\usepackage{listings}
\usepackage{fancybox}
\usepackage{fancyvrb}
\fvset{frame=single, fontsize=\small}

\newcommand{\romannumber}[1]{{\textrm{\uppercase\expandafter{\romannumeral
#1}}}}
\setbeamercolor{dblue}{fg=white,bg=blue!40!black} % for beamercolorbox
\newenvironment{pblock}{\begin{beamercolorbox}[rounded=true,
          shadow=true]{dblue}}{\end{beamercolorbox}}


\graphicspath{{figure/}}

\lstset{
	basicstyle=\footnotesize\ttfamily, % print whole listing footnotesize
	keywordstyle=\footnotesize\ttfamily\color{black}\bfseries, 
	identifierstyle=\footnotesize\ttfamily\color{blue}, 
	commentstyle=\footnotesize\ttfamily\itshape, 
	stringstyle=\footnotesize\ttfamily,
	frame=single, 
	numbers=left, numberstyle=\tiny,
	stepnumber=1, numbersep=10pt,
	showtabs=false, tabsize=4,
	showstringspaces=false,
	breaklines=true, breakatwhitespace=true,
	language=[ISO]C++
}   


%%%%%%%%%%%%%%%%%%%%%%%%%%%%%%%%%%%%%%%%%%%%%%%%%%%%%%%%%%%%%%%%%
%    body                                                       %
%%%%%%%%%%%%%%%%%%%%%%%%%%%%%%%%%%%%%%%%%%%%%%%%%%%%%%%%%%%%%%%%%


\begin{document}

\AtBeginSection[]
{ 
    \begin{frame}<beamer> 
		\frametitle{内容提要} 
		\tableofcontents[currentsection,currentsubsection] 
	\end{frame} 
} 
					
\title{正则表达式}

\author[\href{http://c.pku.edu.cn/}{http://c.pku.edu.cn/}]
{曹东刚\\\href{mailto:caodg@sei.pku.edu.cn}{caodg@sei.pku.edu.cn}}

\institute[北大信科]{Linux程序设计环境 \\
\href{http://c.pku.edu.cn/}{
http://c.pku.edu.cn/}}

\date{}

\titlegraphic{\includegraphics[height=0.17\textwidth]{Overlays/logo.pdf}}

\begin{frame}
	\titlepage
\end{frame}

\section{regexp}

\begin{frame}
\frametitle{基本概念}

\begin{block}{正则表达式(REs: Regular Expressions)}
用于描述一组字符串的表达式, 或称为模式。 英文缩写:
regexp, regex, regxp, regexps, regexes, regexen.

\begin{itemize}
\item 正则表达式由普通字符(例如字符 a 到 z)以及元字符组成.
\end{itemize}
\end{block}

\begin{block}{匹配(matched)}
一个正则表达式和一个字符串匹配,如果该字符串中的字符序列和正则表达式定义的字符序列对应.
\end{block}
\end{frame}


\begin{frame}[fragile]
\frametitle{正则表达式文法}
有多种不同的正则表达式, 文法有差异
    \begin{itemize}
    \item 传统Unix正则表达式, POSIX基本正则表达式(BRE)
    \item POSIX扩展正则表达式(ERE)
    \item Perl-Compatible Regular Expression (PCRE)
    \end{itemize}
\begin{Verbatim}
$ sudo apt-get install pcregrep ack
$ man pcrepattern
$ echo hello | pcregrep -o h
$ echo hello | ack -o h
\end{Verbatim}
\end{frame}

\begin{frame}
\frametitle{正则表达式引擎}

不同的正则表达式引擎对正则表达式有不同的解释
\begin{block} {DFA引擎} 
顺序输入, 速度快, 表达力弱. 最左最长匹配
    \begin{itemize}
    \item awk, egrep, mysql, lex\ldots
    \end{itemize}
\end{block}

\begin{block}{传统NFA引擎} 
	回溯(backtracking), 速度慢, 灵活, 表达力强.  最左匹配
    \begin{itemize}
    \item Perl, Java, .NET, PHP, Python, Ruby, sed, GNU emacs, vi\ldots
    \end{itemize}
\end{block}
\end{frame}

\begin{frame}
\frametitle{正则表达式引擎 (cont.)}

\begin{block}{POSIX NFA引擎} 
匹配成功后仍然尝试所有可能. 最左最长匹配
\end{block}

\begin{block}{混合引擎} 
	DFA和NFA的优点
\begin{itemize}
	\item GNU awk与grep(两个单独的引擎), Tcl (单引擎)
\end{itemize}
\end{block}
\end{frame}

\begin{frame}
\frametitle{NFA引擎}

NFA引擎由正则表达式驱动

\begin{enumerate}
\item 引擎从regex开始, 尝试匹配字符串的首字符. 
  \begin{itemize}
	\item 如果两者不匹配, 则尝试首字符后面的字符.
	\item 如果二者匹配, 引擎会继续读入regex, 并和剩下的字符串匹配. 
  \end{itemize}
\item 如果引擎处理完regex,且字符串匹配, 则认为匹配成功, 放弃其它选择.
\end{enumerate}
\end{frame}

\begin{frame}
\frametitle{NFA引擎 (cont.)}
  若regex有多个选择项, 此时
  \begin{itemize}
	\item NFA引擎将选择其中的一个, 
	  并记录下其它的选择以及当前字符串位置.
	\item 如果随后的匹配失败, 且有选择未尝试,
	  引擎将\alert{回溯}到该选择点,尝试其它选择. 
	\item 如果所有选择都不合法, 则匹配失败, 引擎处理下一个字符.
  \end{itemize}
\end{frame}

\begin{frame}
\frametitle{NFA回溯规则}

\begin{itemize}
\item 总是首先回溯到最近保存的可选点, 如LIFO堆栈
\item 在存在选择(alternation)的情况下, 按给出的顺序依次尝试
\item 对于贪心限定符(greedy quantifier), 首先尝试另一个匹配, 最后选择跳过
\item 对于非贪心限定符(lazy quantifier), 首先选择跳过, 最后尝试其它匹配
\end{itemize}
\end{frame}

\begin{frame}
  \frametitle{NFA回溯规则 (cont.)}
  \begin{itemize}
\item 对于原子组合(atomic grouping)或占有限定符(
possessive quantifier)不保存后向引用.
它们总是作为一个整体被处理.\\
\item 占有限定符会取消回溯
\end{itemize}
\alert{小心给出选择的顺序, 尽量优化以减少回溯}
\end{frame}

\begin{frame}[fragile]
\frametitle{元字符概览}
\footnotesize
\begin{tabular}{|l|p{8cm}|}\hline

字符 & 描述 \\
\hline\hline

. &  匹配任何单个字符(是否匹配换行符依赖于匹配模式及引擎实现) \\
\hline

\textbackslash & 将下一个字符转义. 例: {\verb=\n=} 匹配一个换行符, {\verb=\\=}匹配`\textbackslash'\\
\hline

\textasciicircum & 匹配输入字符串的开始位置. 对于多行匹配,
\textasciicircum 也匹配 `\verb=\n=' 或 `\verb=\r=' 之后的位置 \\
\hline

\$ & 匹配输入字符串的结束位置. 对于多行匹配, \$ 也匹配 `\verb=\n='
或 `\verb=\r=' 之前的位置 \\ \hline

* & 匹配前面的子表达式零次或多次. 例: {ab*} 能匹配 ``a'', ``ab''
\\ \hline

[xy-z]  & 字符集合, 匹配所包含的任意一个字符. 例: {
[abc]} 可以匹配 ``ascii'' 中的 `a'。\\ \hline

[\textasciicircum{}xyz]  & 负值字符集合, 匹配未包含的任意字符. 例: {[\textasciicircum{}abc]} 可以匹配 ``dog'' 中的 `d'。\\
\hline
\end{tabular}
\end{frame}

\begin{frame}[fragile]
\frametitle{元字符概览}
\footnotesize
\begin{tabular}{|l|p{8cm}|}\hline

字符 & 描述 \\
\hline\hline
\verb=\( \)= &
        将\verb=\( \)= 之间的表达式标记为``组''(group),并且保存匹配该表达式的字符(最多可以保存9个),
        用 \textbackslash1 到\textbackslash9的符号来引用\\ \hline

\verb=\{m,n\}= &
        最少匹配 m 次且最多匹配 n 次. 例: {\verb=A[0-9]\{3\}=} 能够匹配A123,A348等,但不匹配A1234.
        {\verb=A[0-9]\{5,6\}=} 匹配A12345, A234345\\ \hline

\verb=\< \>= &
        匹配词的开始和结束. 例: {\verb=\<the=}能够匹配字符串``in the
        day''中的``the'',但不匹配``otherwise''中的``the'' \\
        \hline\hline

\textbar & 将两个匹配条件进行逻辑``或''运算. 例:
{\verb=(him|her)=} 匹配``him''与``her'' \\ \hline

+ & 匹配它之前的那个字符1次或多次; 占有限定符 \\ \hline

? & 匹配它之前的那个字符0次或1次; 非贪心限定符 \\ \hline

\end{tabular}
\end{frame}

\begin{frame}[fragile]
\frametitle{普通字符}

\begin{block} {匹配最先满足regex的字符串}
regex \textbf{is} 匹配``this is a good
toy''中``this''的``is''
\end{block}

\begin{block}{大小写敏感}
regex \textbf{Cat}匹配``Cat'', 不匹配``cat''
\end{block}

需要转义的特殊字符: \verb=[= \verb=\= \verb=^= \verb=$= \verb=.= \verb=|= \verb=?= \verb=*= \verb=+= \verb=(=
\verb=)=

\begin{description}
\item 例: regex {\verb#1+1=2#} 与 {\verb#1\+1=2#}

\item 例: regex {``''}
\end{description}
\end{frame}

\begin{frame}[fragile]
\frametitle{字符集合 ---1}

\begin{itemize}
\item 例: regex {\verb=gr[ae]y=} 匹配 ``gray'' 或者
``grey''

\item 例: regex {\verb=[0-9a-fA-FX]= }
匹配一个16进制数字字母,或字母``X''

\item 例: regex {\verb=[A-Za-z_][A-Za-z_0-9]*= }
程序语言中的标识符

\item 例: regex {\verb=q[^u]=} 匹配``\verb*=Iraq ='',
不匹配``Iraq''
\end{itemize}
\end{frame}

\begin{frame}[fragile]
\frametitle{字符集合 ---2}

\begin{itemize}

\item 字符集合(方括号中)中的特殊字符有``\verb=] - \ ^='', 在特定位置有特殊含义\footnote{转义字符的处理在不同的引擎中有可能有差别,请参考手册说明}.
\item 例: regex {\verb=[+*]=}, {
\verb=[x^]=}, {\verb=[]x]=}, {
\verb=[^]x]=}, {\verb=[-x]=}

\item 字符集合作为一个整体重复匹配. \\
例: regex { \verb=[0-9]+=}

\end{itemize}
\end{frame}

\begin{frame}[fragile]
\frametitle{POSIX预定义字符集合}

\footnotesize

\begin{tabular}{|l|l|l|}\hline

类别 & 含义 & 解释 \\ \hline\hline

\verb=[:upper:]= & \verb=[A-Z]= & 大写字母 \\ \hline

\verb=[:lower:]= & \verb=[a-z]= & 小写字母 \\ \hline

\verb=[:alpha:]= & \verb=[A-Za-z]= & 大小写字母 \\ \hline

\verb=[:alnum:]= & \verb=[A-Za-z0-9]= & 数字与大小写字母 \\
\hline

\verb=[:digit:]= & \verb=[0-9]= & 数字 \\ \hline

\verb=[:xdigit:]= & \verb=[0-9A-Fa-f]= & 十六进制数字 \\ \hline

\verb=[:punct:]= & \verb=[.,!?:...]= & 标点符号 \\ \hline

\verb=[:blank:]= & \verb*=[ \t]= & 空格与制表符 \\ \hline

\verb=[:space:]= & \verb*=[ \t\n\r\f\v]= & 空白字符 \\ \hline

\verb=[:cntrl:]= & & 控制字符 \\ \hline

\verb=[:graph:]= & \verb*=[^ \t\n\r\f\v]= & 可打印字符 \\
\hline

\verb=[:print:]= &  \verb=[^\t\n\r\f\v]= & 可打印字符与空格 \\
\hline

\end{tabular}\\

使用的时候,前面要加括号, 如{\verb=[[:upper:]]=}

\end{frame}

\begin{frame}[fragile]
\frametitle{PCRE预定义字符集合}

\footnotesize

\begin{tabular}{|l|p{8cm}|}\hline

类别 & 含义 \\ \hline\hline

\verb=\s= & 匹配任何空白字符, 等价于\verb*=[ \f\n\r\t\v]= \\
\hline

\verb=\S= & 匹配任何非空白字符, 等价于\verb*=[^ \f\n\r\t\v]= \\
\hline

\verb=\b= & 匹配一个单词边界, 指单词和空格间的位置.例:
{\verb=er\b=} 可以匹配``never'' 中的 `er',但不能匹配 ``verb'' 中的 `er'\\
\hline

\verb=\B= & 匹配非单词边界. 例: {\verb=er\B=}可以
匹配 ``verb'' 中的 `er', 但不能匹配 ``never'' 中的 `er' \\ \hline

\verb=\d= & 匹配一个数字字符, 等价于 \verb=[0-9]=. \\ \hline

\verb=\D= & 匹配一个非数字字符, 等价于 \verb=[^0-9]=. \\ \hline

\verb=\w= & 匹配包括下划线的任何单词字符,
等价于\verb=[A-Za-z0-9_]= \\ \hline

\verb=\W= & 匹配任何非单词字符, 等价于 \verb=[^A-Za-z0-9_]= \\
\hline

\end{tabular}
\end{frame}

\begin{frame}[fragile]
\frametitle{圆点(.)}

缺省情况下, 圆点{\verb=.=} 等价于
{\verb=[^\n]=} 或者 {\verb=[^\n\r]=}
\begin{itemize}

\item 单行模式: 圆点{.} 匹配换行符. PCRE中用{\verb=/s=}控制
\item 多行模式: {\textasciicircum}和{\$}分别扩展匹配字符串中换行符的前后位置.
PCRE中用{\verb=/m=}控制
\item 单行模式只影响圆点, 与多行模式无关
\end{itemize}
\end{frame}


\begin{frame}[fragile]
\frametitle{圆点(.) 示例}

\begin{block}{匹配mm/dd/yy, 允许多种日期分隔符}
\begin{enumerate}
\item {\verb=\d\d.\d\d.\d\d=}
\item {\verb*=\d\d[- /.]\d\d[- /.]\d\d=}
\item {\verb*=[0-1]\d[- /.][0-3]\d[- /.]\d\d=}
\end{enumerate}
\end{block}
\end{frame}

\begin{frame}[fragile]
\frametitle{多行模式示例}
\noindent 找出 grep.txt中的 ``my ice tea''
\begin{verbatim}
$ cat grep.txt
this is my
ice tea.
\end{verbatim}
\pause
\begin{verbatim}
$ pcregrep -M -o 'my\s+ice\s+tea' grep.txt
\end{verbatim}
\end{frame}

\begin{frame}[fragile]
\frametitle{另一个例子}
对下面的输入字符串, 要求匹配一对双引号之间的字符串\\[2ex]
\Ovalbox{Today, he said ``word one'' and ``word two''. What a puzzle.} \\[2ex]

\begin{enumerate}
\item regex {\verb=".*"=} 匹配结果为  ``word one'' and ``word two''  \\
    星号*是贪心限定符
\item regex {\verb*="[^"]*"=} 匹配结果为  ``word one''
\end{enumerate}
\end{frame}

\begin{frame}[fragile]
\frametitle{定位符}

常用定位符: {\verb=^ $ \b \B=}

\begin{block}{验证用户输入合法数字串}
	\begin{itemize}
		\item regex \textbackslash{}d+  (Wrong!) 
		\item regex  \textasciicircum\textbackslash{}d+\$ (Correct)
\end{itemize}
\end{block}

\begin{block}{匹配单字母4}   
	regex \textbackslash{}b4\textbackslash{}b 
\end{block}
\end{frame}

\begin{frame}[fragile]
\frametitle{选择符}
选择符~\alert{\textbar}~的优先级最低

\begin{itemize}
\item reg {\verb=\b(cat|dog)\b=} : 匹配单词cat或dog 

\begin{itemize} \item 比较 reg {\verb=\bcat|dog\b=}
\end{itemize}

\item
输入字符串为``SetValue'', \\
regex 为 {\verb=Get|GetValue|Set|SetValue=}
    \begin{itemize}
    \item 比较: regex {\verb=Get|GetValue|SetValue|Set=}
    \item 比较: regex {\verb=Get(Value)?|Set(Value)?=}
    \end{itemize}
\end{itemize}
\end{frame}


\begin{frame}[fragile]
\frametitle{数量限定符}
\begin{itemize}
\item 数量限定符{\verb=? * +=}以及{\verb={m,n}=} 缺省情况下是``贪心''的

\item 例: 找到下面字符串中的HTML标识符, \\[2ex]
	\Ovalbox{This is a <EM>first</EM> test} \\

    \begin{itemize}
    \item regexp {\verb=<.+>=} 匹配的结果是: 
$<$EM$>$first$<$/EM$>$
    \end{itemize}
\end{itemize}

\end{frame}

\begin{frame}[fragile]
  \frametitle{非贪心限定符}

  \begin{itemize}
\item 将数量限定符改为``非贪心''(Laziness):
将问号\textbf{?}放到表达式的后面
    \begin{itemize}
    \item 对上面例子应用regexp {\verb=<.+?>=}
    \item 比较负值运算: regexp {\verb=<[^>]+>=}
    \item 对于输入字符串``1234567 122'', 比较 regexp
{\verb=[1-9][0-9]{2,4}=} 和 {\verb=[1-9][0-9]{2,4}?=}
    \end{itemize}
\end{itemize}

\end{frame}

\begin{frame}[fragile]
\frametitle{组合与后向引用}

一对圆括号()将其中的所有正则表达式组合为一个子表达式,
并创建一个后向引用(backreference).

\begin{itemize}
\item regex {\verb=Set(Value)?=} 匹配 ``Set'' 或者
``SetValue'', 对于前者, 后向引用的值为空; 对于后者,
后向引用的值为``Value'' 
\item 例: regex {\verb=([a-c])x\1=} 匹配:  ``axa'',
``bxb'', ``cxc'' 
\end{itemize}
\end{frame}

\begin{frame}[fragile]
\frametitle{组合与后向引用 (cont.)}

更多例子
\begin{itemize}

\item 对 ``\verb=Testing <B><I>bold italic</I></B> text=''\\
  应用 regex {\verb=<([A-Z][A-Z0-9]*)[^>]*>.*?</\1>=}
    \begin{itemize}
    \item 结果为: ``\verb=<B><I>bold italic</I></B>='' 
    \end{itemize}

\item 例: 对字符串``cab'', 比较regex {\verb=([abc]+)=} 和 
	regex { \verb=([abc])+=}
    \begin{itemize}
    \item 前者后向引用为``cab'', 后者为``b''(后面的结果覆盖前面的)
    \end{itemize}
\end{itemize}
\end{frame}

\begin{frame}[fragile]
    \frametitle{断言}
\noindent 问题: 找到字符串 \verb~<td>Nice</td>~ 中td tag的内容. \\
\noindent 方法: 前向(look-ahead)与后向(look-behind)断言.
\begin{block}{前向断言}
    \begin{itemize}
        \item 肯定断言 \verb~(?=)~
            \begin{itemize}
                \item 例: \verb~\w+(?=;)~
    \end{itemize}
        \item 否定断言 \verb~(?!)~
            \begin{itemize}
                \item 例: \verb~ foo(?!bar)~ 
            \end{itemize}
    \end{itemize}
\end{block}
\end{frame}

\begin{frame}[fragile]
    \frametitle{断言--- cont.}
\begin{block}{后向断言}
    \begin{itemize}
        \item 肯定断言 \verb~(?<=)~
            \begin{itemize}
                \item 例: \verb~(?<=linu)X~
    \end{itemize}
        \item 否定断言 \verb~(?<!)~
            \begin{itemize}
                \item 例: \verb~ (?<!foo)bar~ 
            \end{itemize}
    \end{itemize}
\end{block}
\verb~<td>Nice</td>~的tag内容的匹配: \\
\verb~(?<=<td>).*?(?=</td>)~
\end{frame}

\begin{frame}[fragile]
\frametitle{占有限定符 possessive quantifier}
\noindent 取消贪心匹配的回溯, 数量限定符后加加号 +

\begin{Verbatim}
$ echo ababacd | ack -o '(ab)*a'
$ echo ababacd | ack -o '(ab)*?a'
$ echo ababacd | ack -o '(ab)*+a'
$ echo aaaaaaa | ack -o 'a?+a'
$ echo aaaaaaa | ack -o 'a+a'
$ echo aaaaaaa | ack -o 'a++a'
\end{Verbatim}

\end{frame}


\begin{frame}[fragile]
\frametitle{优先级顺序}

\begin{tabular}{|l|p{4cm}|} \hline

操作符 & 描述 \\ \hline\hline

\textbackslash & 转义符 \\ \hline

\verb=(), []= & 组合符和集合符 \\ \hline

\verb=*, +, ?, {n}, {n,}, {n,m}= & 数量限定符 \\ \hline

\verb=^, $= & 定位符和顺序 \\ \hline

\verb=|= & 逻辑``或''操作 \\ \hline

\end{tabular}
\end{frame}

\section{grep}

\begin{frame}[fragile]
	\frametitle{grep}

\alert{grep}(全局正则表达式), 允许
对文本文件进行模式查找。如果找到匹配模式,
\alert{grep}打印包含模式的所有行

{grep}有三种变型
\begin{itemize}
\item grep: 标准grep, 支持基本正则表达式
\item egrep: 扩展grep, 支持基本及扩展的正则表达式
\item fgrep: 快速grep, 允许查找字符串而不是一个模式
\end{itemize}

\Ovalbox{ack 被设计为取代99\%的grep使用, ack支持PCRE}

\end{frame}


\begin{frame}[fragile]
	\frametitle{grep (cont.)}
{grep}命令格式:\\
grep [选项] 正则表达式 [文件]

\begin{Verbatim}
$ echo abcdabcd | grep -o -P '(ab)*a' # pcre grep
\end{Verbatim}

\end{frame}

\begin{frame}[fragile]
\frametitle{单引号与双引号}

shell在处理命令时, 要对输入进行解释. 为了把参数正确传递给程序,
有时候需要阻止shell扩展参数中的特殊字符,
这可通过单引号或双引号实现.

\begin{itemize}
\item 双引号: 只对输入字符串中的\verb=$ `` \=进行扩展\\[2em]
\begin{Verbatim}
$ BOY=boy; echo "The $BOY did well"
$ echo "The $DAY is `date`"
\end{Verbatim}

\end{itemize}
\end{frame}

\begin{frame}[fragile]
\frametitle{单引号与双引号(cont.)}
\begin{itemize}
\item 单引号: 对输入字符串不做任何特殊处理\\[1em]
\begin{Verbatim}
caodg@debian:~$ BOY=boy; echo 'The $BOY did well'
\end{Verbatim}

\item 输入模式参数给{grep}时, 用单括号将模式括起来\\[1em]
\begin{Verbatim}
caodg@debian:~$ grep '[ab]*|45$' input.txt
\end{Verbatim}
\end{itemize}
\end{frame}

\begin{frame}[fragile]
\frametitle{grep示例}

\begin{itemize}
\item 在当前目录下的*.c文件中查找字符串``printf''
(不分大小写)\\
\begin{Verbatim}
caodg@debian:~$ grep -i 'printf' *.c
\end{Verbatim}

\item 统计文件main.c中包含字符串``printf''的总行数\\
\begin{Verbatim}
caodg@debian:~$ grep -c 'printf' main.c
\end{Verbatim}

\item 列出main.c中包含字符串``printf''的每一行\\
\begin{Verbatim}
caodg@debian:~$ grep -n 'printf' main.c
\end{Verbatim}
\end{itemize}
\end{frame}


\begin{frame}[fragile]
\frametitle{grep示例 (cont.)}

\begin{itemize}
\item 列出main.c中不包含字符串``printf''的每一行\\[1em]
\begin{Verbatim}
caodg@debian:$ grep -v 'printf' main.c
\end{Verbatim}

\item
递归搜索目录worm下的包含字符串``trojan''的文本文件\\[1em]
\begin{Verbatim}
grep -r --binary-files=without-match 'trojan' worm
\end{Verbatim}

\end{itemize}
\end{frame}

\section{find}

\begin{frame}[fragile]
\frametitle{find}

\alert{find} 查找文件系统中具有特定特征的文件\\
命令格式: \verb=find [路径] [表达式]=

\begin{itemize}
\item 查找/home目录下所有的core文件\\
\begin{Verbatim}
caodg@debian:~$ find /home -name core -print
\end{Verbatim}

注意: 尽量输出相对路径, 当备份时可以在临时目录恢复.
\end{itemize}
\end{frame}


\begin{frame}[fragile]
\frametitle{find---示例}

\begin{itemize}
\item 查找/home目录下文件名包含``core''的文件\\
\begin{Verbatim}
caodg@debian:~$ find /home -name '*core*' -print
\end{Verbatim}

\item 查找属主帐户被删除的文件\\
\begin{Verbatim}
caodg@debian:~$ find /var -nouser 
\end{Verbatim}
\end{itemize}
\end{frame}

\begin{frame}[fragile]
\frametitle{查找特定类型的文件}

\begin{itemize}
\item 查找/home目录下所有的目录文件
\begin{Verbatim}
caodg@debian:~$ find /home -type d 
\end{Verbatim}
{\small
\begin{tabular}{c@{\hspace{2cm}}l} \hline
类型 & 描述 \\ \hline

f & 普通文件 \\
d & 目录 \\
b & 块设备文件 \\
c & 字符设备文件 \\
l & 符号链接文件 \\
p & 管道文件 \\ \hline

\end{tabular}}
\end{itemize}
\end{frame}

\begin{frame}[fragile]
\frametitle{按大小查找文件}

\begin{itemize}
\item 查找/home目录下大于100M的文件\\
\begin{Verbatim}
caodg@debian:~$ find /home -size +100M 
\end{Verbatim}
{\small
\begin{tabular}{c@{\hspace{2cm}}l} \hline
类型 & 描述 \\ \hline

b & 512 byte磁盘块(default) \\
c & 字节 \\
w & 双字节 \\
k & kilobyte \\
M & megabyte \\
G & gigabyte \\ \hline
\end{tabular}}

\item 注意: 数字前面加加号, 表示大于给定值; 减号, 表示小于给定值;
什么都不加表示等于
\end{itemize}
\end{frame}


\begin{frame}[fragile]
\frametitle{按时间查找文件}

\begin{itemize}
\item 查找/home目录下最近5天内内容修改过的文件\\
\begin{Verbatim}
caodg@debian:~$ find /home -mtime -5 
\end{Verbatim}

\item 查找/home目录下5天前访问过的文件\\
\begin{Verbatim}
caodg@debian:~$ find /home -atime +5 
\end{Verbatim}

\item 查找/home目录下最近5天内状态改变过的文件\\
\begin{Verbatim}
caodg@debian:~$ find /home -ctime -5 
\end{Verbatim}

\item 查找当前目录下比a.txt新的文件\\
\begin{Verbatim}
caodg@debian:~$ find . -newer a.txt 
\end{Verbatim}

\end{itemize}
\end{frame}

\begin{frame}[fragile]
\frametitle{选项组合}

\begin{itemize}
\item 查找/home目录下最近5天内内容修改过的普通文件\\
\begin{Verbatim}
~$ find /home -type f -mtime -5 
\end{Verbatim}


\item 查找当前目录下比a.txt新, 但比b.txt老的文件\\
\begin{Verbatim}
~$ find . -newer a.txt ! -newer b.txt
\end{Verbatim}

\item 查找core文件或者tmp目录\\
\begin{Verbatim}
~$ find . -name core -o \( -type d -name tmp \)
\end{Verbatim}
\begin{itemize}
\item 注意空格: ``\verb*=\( expression \)=''
\end{itemize}

\end{itemize}
\end{frame}

\begin{frame}[fragile]
\frametitle{对找到的文件执行动作}

\begin{itemize}
\item 删掉/home目录下的所有core文件\\
\begin{Verbatim}
~$ find /home -name core -exec rm -f {} \;
\end{Verbatim}

\begin{Verbatim}
~$ find /home -name core -delete
\end{Verbatim}


\item 将所有tgz文件移动到/tmp目录下并确认\\
\begin{Verbatim}
~$ find /home -name "*.tgz" -ok mv {} /tmp \;
\end{Verbatim}
\end{itemize}

\end{frame}

\end{document}
