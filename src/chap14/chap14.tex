\documentclass{beamer}

\usepackage[nofonts]{ctex}
\setCJKmainfont[ItalicFont={Kaiti SC}]{Kaiti SC}
%\setCJKmainfont[ItalicFont={AR PL KaitiM GB}]{AR PL KaitiM GB}%
%\setCJKsansfont{WenQuanYi Zen Hei}% 文泉驿的黑体

\mode<beamer>
{
    \useinnertheme{rounded}
    \useoutertheme{split}
    \usecolortheme{rose}
    \usecolortheme{seahorse}

	\expandafter\def\expandafter\insertshorttitle\expandafter{%
	\insertshorttitle\hfill%
	\insertframenumber\,/\,\inserttotalframenumber}
}

\mode<handout>
{
	\usetheme{default}
	\usepackage{pgfpages}
	\pgfpagesuselayout{4 on 1}[a4paper,landscape,border shrink=5mm]
}


\usepackage{amsmath,latexsym,amssymb,amsfonts,amsbsy}
\usepackage{graphicx}
\usepackage{hyperref}
\usepackage{listings}
\usepackage{fancyvrb}
\fvset{frame=single,fontsize=\small}

\newcommand{\romannumber}[1]{{\textrm{\uppercase\expandafter{\romannumeral
#1}}}}
\setbeamercolor{dblue}{fg=white,bg=blue!40!black} % for beamercolorbox
\newenvironment{pblock}{\begin{beamercolorbox}[rounded=true,
              shadow=true]{dblue}}{\end{beamercolorbox}}


\graphicspath{{figure/}}

\lstset{
	basicstyle=\footnotesize, % print whole listing footnotesize
	keywordstyle=\color{black}\bfseries, 
	identifierstyle=\color{blue}, 
	commentstyle=\itshape, 
	stringstyle=\ttfamily,
	frame=single, 
	numbers=left, numberstyle=\tiny,
	stepnumber=1, numbersep=10pt,
	showtabs=false, tabsize=4,
	showstringspaces=false,
	breaklines=true, breakatwhitespace=true,
	language=[ISO]C++
}   


%%%%%%%%%%%%%%%%%%%%%%%%%%%%%%%%%%%%%%%%%%%%%%%%%%%%%%%%%%%%%%%%%
%    body                                                       %
%%%%%%%%%%%%%%%%%%%%%%%%%%%%%%%%%%%%%%%%%%%%%%%%%%%%%%%%%%%%%%%%%


\begin{document}

\AtBeginSection[]
{ 
    \begin{frame}<beamer> 
		\frametitle{内容提要} 
		\tableofcontents[currentsection,currentsubsection] 
	\end{frame} 
} 
					
\title{开源软件的经济模式}

\author[\href{http://c.pku.edu.cn/}{http://c.pku.edu.cn/}]
{曹东刚\\\href{mailto:caodg@sei.pku.edu.cn}{caodg@sei.pku.edu.cn}\\
\href{http://c.pku.edu.cn/}{
http://c.pku.edu.cn/}}

\institute{Linux程序设计环境}
\date{}

\titlegraphic{\includegraphics[height=0.17\textwidth]{Overlays/logo.pdf}}

\begin{frame}
	\titlepage
\end{frame}

\begin{frame}
\frametitle{The emerging economic paradigm of open source}
By Bruce Perens \\
\href{http://firstmonday.org/htbin/cgiwrap/bin/ojs/index.php/fm/article/viewArticle/1470/1385}{First
Monday, Special Issue \#2: Open Source — 3 October 2005}

\end{frame}

\begin{frame}
	\frametitle{开源是有经济模式的}
	\begin{itemize}
		\item ESR 的 ``The Cathedral and bazaar''带来的误解: gift economy
		\item 当今红火的开源运动背后必有商业模式
	\end{itemize}
\end{frame}

\begin{frame}
	\frametitle{微软的模式}
	从厂商的角度: 你准备如何成为下一个微软? \\
	微软模式:仅对少部分单独售卖的软件(1/3)适用
\end{frame}

\begin{frame}
	\frametitle{未被满足的需求}
	\begin{itemize}
		\item RMS: 提出程序员的哲学理想---非商业人士的.
		\item 微软: 垄断, 质量差, 用户无选择.
		\item 开源: 让软件质量更高
			\begin{itemize}
				\item 企业用户直接参与开发
				\item 企业互相合作
				\item 鼓励竞争, 消除垄断
			\end{itemize}
	\end{itemize}
\end{frame}

\begin{frame}
	\frametitle{正视未曾实现的需求}
	\begin{itemize}
		\item 年轻大学生凭兴趣开发的操作系统统治了企业计算市场
		\item IBM将重心从AIX转移到开源Linux
		\item 微软首次受到业余程序员们的重大挑战
	\end{itemize}
\end{frame}

\begin{frame}
	\frametitle{寻找经济模式}
	\begin{block}{微软的成功模式}
	使商业伙伴可在微软的软件协助下更好发展, 如果离开微软他们甚至无法开展自己的业务.
	\end{block}
\end{frame}

\begin{frame}
	\frametitle{商业中的软件}
	\begin{itemize}
		\item 区分业务的成本中心和利润中心
			\begin{itemize}
				\item 软件通常在成本中心中
			\end{itemize}
		\item 软件是大多数商业业务的使能技术
			\begin{itemize}
				\item 差异化 vs 无差异化
			\end{itemize}
	\end{itemize}
\end{frame}

\begin{frame}
	\frametitle{差异化 vs 大众化}
	\begin{itemize}
		\item 差异化: 核心竞争力, 如Amazon的推荐系统
			\begin{itemize}
				\item 竞争对手可否获得该产品
				\item 用户是否可见到软件的效果
			\end{itemize}
		\item 大众化: 商业业务中90\%以上的软件
	\end{itemize}
\end{frame}

\begin{frame}
	\frametitle{初步结论}
	重点关注差异化、能提升业务的软件. \\
	前提:能够成功找到自己业务中的差异化部分.
\end{frame}

\begin{frame}
	\frametitle{NIH综合症}
	NIH(Not Invented Here): 极大增加了业务成本 \\
	\begin{block}{对策: 将非差异化部分开源}
		风险和成本分摊. 集中关注核心的差异化部分.
	\end{block}
\end{frame}

\begin{frame}
	\frametitle{软件开发的经济模式}
\begin{itemize}
	\item 零售
	\item 内部开发与外包
	\item 非开源性质的联盟
	\item 开源
\end{itemize}
\end{frame}

\begin{frame}
	\frametitle{零售模式}
	\begin{itemize}
		\item 只占软件开发中的30\%
		\item 必须在产品成熟时才可推向市场
		\item 由一家开发商单独承担极高的开发成本和市场风险
		\item 售价的10\%用于产品自身研发
		\item 只有在规模市场时才可成功
		\item 不鼓励创新: WEB的例子
	\end{itemize}
\end{frame}


\begin{frame}
	\frametitle{内部开发与外包模式}
	\begin{itemize}
		\item 内部开发: 客户自己雇佣程序员直接开发软件. 
		\item 外包开发: 将开发任务承包给外部开发机构.
		\item 客户对软件可控, 但风险和成本自担
		\item 适合开发差异化软件, 不适合非差异化软件
		\item 开发效率: 50\% - 80\% 
		\item 承包开发者的问题
	\end{itemize}
\end{frame}

\begin{frame}
	\frametitle{ 非开源性质的联盟 }

	效率低下, 对差异化和非差异化软件的开发都不适合.
	
\end{frame}

\begin{frame}
	\frametitle{开源模式}
	\begin{itemize}
		\item 首先内部开发, 发布
		\item \emph{别人发现有用}
		\item 改善扩展
			\begin{itemize}
				\item 新代码要及时提交给原开发者, 否则难以维护
			\end{itemize}
		\item 形成开发社区是关键
		\item 商业公司可通过开源软件提升自己的差异化业务
	\end{itemize}
\end{frame}

\begin{frame}
	\frametitle{谁在为开源软件贡献}
	\begin{itemize}
		\item 志愿者
		\item Linux厂商
		\item 以开源软件为唯一业务的公司
		\item 以开源软件为使能技术的公司
		\item 服务商业
		\item 终端用户
		\item 政府
		\item 科研机构
	\end{itemize}
\end{frame}

\begin{frame}
	\frametitle{ 以开源软件为唯一业务的公司 }

	\begin{itemize}
		\item 开源和商业混合许可: MySQL BB
		\item 开源软件, 和商业插件: Sendmail, Linux+DB2
		\item 纯粹开源, 和服务支持
	\end{itemize}
\end{frame}

\begin{frame}
	\frametitle{以开源软件为使能技术的公司}
	以硬件厂商为代表
\end{frame}

\begin{frame}
	\frametitle{服务商业}
	将软件功能以服务的形式对外提供, 规避``分发问题''. \\
	可能会受制于未来改进的GPL协议.
\end{frame}

\begin{frame}
	\frametitle{开源软件如何做市场}
    \begin{pblock}
        ``达尔文主义''的自然进化\\
    \end{pblock}
	\begin{itemize}
		\item 自由竞争
		\item 50个用户以上的首先胜出
		\item 形成开发社区的持续发展
		\item 少独裁专制, 多自由民主
	\end{itemize}
\end{frame}

\begin{frame}
	\frametitle{对程序员的影响}
	\begin{itemize}
		\item 对程序员的需求不会变少
		\item 工资可能稍有影响, 但无关大局
	\end{itemize}
\end{frame}

\begin{frame}
	\frametitle{搭便车的问题}
	这其实不是个问题
\end{frame}

\begin{frame}
	\frametitle{能生产开源的汽车吗?}
	性质不同, 也许将来可以.
\end{frame}

\end{document}
