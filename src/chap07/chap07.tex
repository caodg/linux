\documentclass[compress]{beamer}

\usepackage[nofonts]{ctex}
\setCJKmainfont[ItalicFont={Kaiti SC}]{Kaiti SC}%
%\setCJKsansfont{Heiti SC}% 文泉驿的黑体

\mode<beamer>
{
    \setbeamercovered{transparent}

    \useinnertheme{rounded}
    %\useoutertheme{miniframes}
    \useoutertheme{split}
    %\usecolortheme{orchid}
    %\usecolortheme{whale}
    %\usecolortheme{lily}
    \usecolortheme{rose}
    \usecolortheme{seahorse}

    %\usetheme{Warsaw}
	%\useoutertheme[footline=institutetitle]{miniframes}

	\expandafter\def\expandafter\insertshorttitle\expandafter{%
	\insertshorttitle\hfill%
	\insertframenumber\,/\,\inserttotalframenumber}
}

\mode<handout>
{
	\usetheme{default}
	\usepackage{pgfpages}
	\pgfpagesuselayout{4 on 1}[a4paper,landscape,border shrink=5mm]
}


\usepackage{amsmath,latexsym,amssymb,amsfonts,amsbsy}
\usepackage{graphicx}
\usepackage{hyperref}
\usepackage{listings}
\usepackage{fancyvrb}
\fvset{frame=single,fontsize=\small}

\newcommand{\romannumber}[1]{{\textrm{\uppercase\expandafter{\romannumeral
#1}}}}
\setbeamercolor{dblue}{fg=white,bg=blue!40!black} % for beamercolorbox
\newenvironment{pblock}{\begin{beamercolorbox}[rounded=true,
                      shadow=true]{dblue}}{\end{beamercolorbox}}

\graphicspath{{figure/}}

\lstset{
	basicstyle=\footnotesize\ttfamily, % print whole listing footnotesize
	keywordstyle=\footnotesize\ttfamily\color{black}\bfseries, 
	identifierstyle=\footnotesize\ttfamily\color{blue}, 
	commentstyle=\footnotesize\ttfamily\itshape, 
	stringstyle=\footnotesize\ttfamily,
	frame=single, 
	numbers=left, numberstyle=\tiny,
	stepnumber=1, numbersep=10pt,
	showtabs=false, tabsize=4,
	showstringspaces=false,
	breaklines=true, breakatwhitespace=true,
	language=sh
}   


%%%%%%%%%%%%%%%%%%%%%%%%%%%%%%%%%%%%%%%%%%%%%%%%%%%%%%%%%%%%%%%%%
%    body                                                       %
%%%%%%%%%%%%%%%%%%%%%%%%%%%%%%%%%%%%%%%%%%%%%%%%%%%%%%%%%%%%%%%%%


\begin{document}

\AtBeginSection[]
{ 
    \begin{frame}<beamer> 
		\frametitle{内容提要} 
		\tableofcontents[currentsection,hideallsubsections] 
	\end{frame} 
} 
					
\title{文本处理}

\author[\href{http://c.pku.edu.cn/}{http://c.pku.edu.cn/}]
{曹东刚\\\href{mailto:caodg@sei.pku.edu.cn}{caodg@sei.pku.edu.cn}}

\institute{Linux程序设计环境 \\
\href{http://c.pku.edu.cn/}{
http://c.pku.edu.cn/}}

\date{}

\titlegraphic{\includegraphics[height=0.17\textwidth]{Overlays/logo.pdf}}

\begin{frame}
	\titlepage
\end{frame}


\section{小工具}

\subsection{tr}

\begin{frame}[fragile]
\frametitle{tr}
\emph{tr} 用于从标准输入中通过替换或删除操作进行字符转换\\
\emph{tr}命令格式\\
\verb~tr [OPTION] SET1 [SET2]~\\
常用OPTION
\begin{itemize}
\item -s : 删除 SET1 中重复出现的字符序列
\item -d : 删除 SET1 中所有字符
\item -c : 求 SET1 的补集
\end{itemize}
\end{frame}

\begin{frame}[fragile]
\frametitle{字符范围}
\begin{tabular}{|l@{\hspace{0.5cm}}l|}
\hline

\verb=[c1-c2]= & ASCII字符c1和c2之间的字符\\
\verb=[c1c2]= &  ASCII字符c1和c2\\
\verb=/oct= & 三位8进制数, 对应ASCII字符 \\
\verb=[O*n]= & 字符O重复出现n次\\ \hline
\end{tabular}

\end{frame}


\begin{frame}[fragile]
\frametitle{示例}

\begin{itemize}
\item 去除重复出现的字符 \\
\begin{Verbatim}
tr -s "[a-z]"
\end{Verbatim}

\item 删除空行 \\
\begin{Verbatim}
tr -s "\n"
\end{Verbatim}

\item 大小写转换\\
\begin{Verbatim}
tr "[a-z]" "[A-Z]"
\end{Verbatim}
\end{itemize}

\end{frame}

\begin{frame}[fragile]
\frametitle{示例 (cont)}

\begin{itemize}
\item 将DOS格式(回车\verb~\r~、换行\verb~\n~)转换为Unix格式(以\verb~\n~换行)\\
\begin{Verbatim}
tr -d "\r"
\end{Verbatim}

\item 找到文件中的所有单词, 每行一个输出
\begin{Verbatim}
tr -cs "[a-zA-Z]" '\n'
\end{Verbatim}
\end{itemize}

\end{frame}

\subsection{cut/paste}

\begin{frame}[fragile]
\frametitle{cut}
\emph{cut}用来从标准输入或文本文件中抽取数据列或域\\
\emph{cut}命令格式\\
\verb~cut [OPTION] [FILES]~\\
常用OPTION
\begin{itemize}
\item -c LIST : 指定剪切字符数. LIST如下单表达式或由逗号分隔的表达式组合:
N : 第N个字符; N--M : 第N到第M个字符; N-- : 从第N个字符到行尾

\item -d delimiter-char : 指定非TAB分隔字符
\item -f LIST : 指定剪切的域, LIST形式同-c
\end{itemize}

\end{frame}

\begin{frame}[fragile]
\frametitle{示例}

\begin{itemize}
\item 从文件/etc/passwd中剪切用户名及用户主目录 \\
passwd文件格式:\\
\verb~ftp:x:103:65534::/home/ftp:/bin/false~\\
\begin{Verbatim}
cut -d : -f1,6 /etc/passwd
\end{Verbatim}

\item 显示当前谁在使用系统\\
\begin{Verbatim}
who |  cut -c 1-8
\end{Verbatim}

\end{itemize}

\end{frame}

\begin{frame}[fragile]
\frametitle{paste}
\emph{cut} 用来从文本文件或标准输入中抽取数据列或者域, \emph{paste} 将按行将不同文件行信息放在一行\\
\emph{paste}命令格式\\
\verb~paste [OPTION] [FILES]~\\
常用OPTION
\begin{itemize}
\item -d delimiter-char : 指定非TAB分隔字符
\item -s : 将每个文件粘贴为一行而不是逐行粘贴
\end{itemize}
\end{frame}

\begin{frame}[fragile]
\frametitle{示例}

\begin{itemize}
\item 将文件a.txt和b.txt合并 \\
\begin{Verbatim}
paste a.txt b.txt
\end{Verbatim}

\item 从管道输入, 每行显示多列\\
利用选项``--'', 意即对每一个``--'', 从标准输入中读一次数据。\\
\begin{Verbatim}
ls | paste - - - -
\end{Verbatim}

\end{itemize}

\end{frame}

\subsection{sort/uniq}

\begin{frame}
\frametitle{sort}
\begin{itemize}
\item \emph{sort} 命令将许多不同的域按不同的列顺序分类
\item sort假定工作文件已经被分过类
    \begin{itemize}
    \item 文件是结构化的, 符合一定的模式, 如/etc/passwd文件
    \item 通常情况用\emph{awk}将文件格式化输出后, 再用\emph{sort}进行排序
    \end{itemize}
\item 缺省情况下, \emph{sort} 以一系列空格或TAB为分隔符

\end{itemize}
\end{frame}

\begin{frame}[fragile]
\frametitle{示例}

\begin{itemize}
\item 基本排序, 缺省按照第一域按字典进行分类 \\
\begin{Verbatim}
sort -t: /etc/passwd
\end{Verbatim}

\item 逆向排序, 对第三列按数字排序 \\
\begin{Verbatim}
sort -t: -r +2n /etc/passwd
sort -t: -r -k3n /etc/passwd
\end{Verbatim}

\item 分类排序, 先对passwd文件第三列排序, 再对第七列排序\\
\begin{Verbatim}
sort -t: -k3 -k7 /etc/passwd
\end{Verbatim}

\item 将两个已分类文件合并\\
\begin{Verbatim}
sort -m +0 patch.txt sorted.txt
\end{Verbatim}

\end{itemize}

\end{frame}

\begin{frame}[fragile]
\frametitle{uniq}
\begin{itemize}
\item  \emph{uniq} 用来从一个文本文件中去除或禁止重复行
\item 一般 \emph{uniq} 假定文件已分类, 并且结果正确
\end{itemize}
\emph{uniq}命令格式\\
\verb~uniq [OPTION] [INPUT]~\\
常用OPTION
\begin{itemize}
\item -u : 只显示不重复行
\item -d : 只显示有重复数据行, 每种重复行只显示其中一行
\item -c : 打印每一重复行出现次数
\end{itemize}

\end{frame}

\begin{frame}[fragile]
    \frametitle{统计文件中单词出现频率}

    \begin{itemize}
    \item <+-> 找所有的单词
\begin{Verbatim}
    grep -o -P '\b[a-zA-Z]+\b'    
\end{Verbatim}
    \item <+-> 统计
\begin{Verbatim}
    sort | uniq -c   
\end{Verbatim}
\end{itemize}
\end{frame}

\subsection{join}

\begin{frame}
\frametitle{join}
\begin{itemize}
\item \emph{join} 用来将来自两个分类文本文件的行连在一起, 类似SQL语言的join命令
\item 文本文件中的域通常由空格或tab键分隔, 也可以指定其他的域分隔符
\item 一些系统要求使用 \emph{join} 时文件域要少于20; 如果域大于20, 应使用DBMS系统
\item 为有效使用\emph{join}, 需分别将输入文件分类
\end{itemize}

\end{frame}


\begin{frame}[fragile]
\frametitle{示例: 数据文件}

数据文件 name.dat
\begin{Verbatim}
s00348228 1034 LiRuihang 
s00348230 1036 MaJianzhu 
s00348281 1035 RaoXiangrong 
s00348282 1037 ChengZhiwen 
\end{Verbatim}

数据文件 score.dat
\begin{Verbatim}
s00348228 10  20 
s00348229 8   18
s00348231 9   19 
s00348282 7   20
\end{Verbatim}


\end{frame}

\begin{frame}[fragile]
\frametitle{示例}

\begin{itemize}
\item 连接两个文件 \\
例: join name.dat score.dat 结果:\\
\begin{tabular}{lllll}
s00348228 & 1034 & LiRuihang & 10 & 20\\
s00348282 & 1037 & ChengZhiwen & 7 & 20\\
\end{tabular} 

\item 指定连接域 \\
例: join -1 1 -2 2 name.dat score.dat

\item 指定输出域\\
例: join -o1.1,2.2,2.3
结果:\\
\begin{tabular}{lll}
s00348228 & 10 & 20\\
s00348282 & 7 & 20\\
\end{tabular}

\end{itemize}

\end{frame}

\subsection{head/tail}

\begin{frame}[fragile]
	\frametitle{head和tail}
head和tail分别读取文件的前几行与后几行. 这在要读取的文件(通常是log)很大(如200M), 要读取的内容恰好在文件的开始/结束位置时很好用. 

\begin{Verbatim}
tail -30 /var/log/syslog
\end{Verbatim}
\end{frame}

\subsection{xargs}

\begin{frame}[fragile]
    \frametitle{xargs 从标准输入重建并执行命令行}
\begin{Verbatim}
$cat t.txt 
1 2 3 
4 5 6
\end{Verbatim}
\begin{Verbatim}
$cat t.txt | xargs echo
1 2 3 4 5 6
\end{Verbatim}
\begin{Verbatim}
$cat t.txt | xargs -n 2 
1 2 
3 4 
5 6
\end{Verbatim}
\end{frame}

\begin{frame}[fragile]
    \frametitle{xargs 示例}
\begin{Verbatim}
$echo "foo,bar,baz" | xargs -d, -L 1 echo
\end{Verbatim}
\pause
\begin{Verbatim}
$ls | xargs -t -L 4 echo
\end{Verbatim}
\pause
\begin{Verbatim}
$xargs -a foo -d, -L 1 echo
\end{Verbatim}
\pause
\begin{Verbatim}
$find . -type f -name "*.c" -print0 | xargs -0 wc -l
\end{Verbatim}
\pause
\noindent 避免 ``Too Many Arguments''问题
\begin{Verbatim}
$find . -type d -name ".svn" -print | xargs rm -rf
\end{Verbatim}
\end{frame}

\subsection{其他}

\begin{frame}[fragile]
	\frametitle{tee}
tee可以从标准输入读取数据,并数据输出到标准输出上,并输出到多个文件中.

\begin{Verbatim}
ls -l | tee ls.txt lt.txt
\end{Verbatim}
\end{frame}

\begin{frame}[fragile]
	\frametitle{iconv与convmv}
改变文本文件内容的编码.

\begin{Verbatim}
iconv -f utf8 -t gbk
\end{Verbatim}

改变文本文件名的编码: convmv \\

可能需要首先探查文件编码: enca

\end{frame}

\begin{frame}[fragile]
\frametitle{split}
\emph{split} 用来将大文件分割成小文件, 缺省以1000行为单位进行分割.
通过 -b 参数可以指定分割单位, 如-b 50m指定以50M为单位

\begin{itemize}
\item 例: 文件\verb~debian-installer.iso~为110M, 现在将其分解为60M的小文件\\
split -b 60m debian-installer.iso\\
生成文件名为xaa xab的两个文件

\item 例: 将文件xaa xab复原\\
cat xaa xab $>$ x.iso
\end{itemize}

\end{frame}

\subsection{示例}

\begin{frame}[fragile]
\frametitle{综合示例: 统计出现频率最高的单词}
\begin{verbatim}
#!/bin/sh
# 统计文本流中出现频率最高的前n个单词列表
# usage: wf [n]
\end{verbatim}
\pause
\begin{verbatim}
tr -cs A-Za-z '\n'      | 
\end{verbatim}
\pause \vspace{-5ex}
\begin{verbatim}
 tr A-Z a-z             | 
\end{verbatim}
\pause \vspace{-5ex}
\begin{verbatim}
   sort                 | 
     uniq -c            | 
\end{verbatim}
\pause \vspace{-5ex}
\begin{verbatim}
       sort -k1nr -k2   |
\end{verbatim}
\pause \vspace{-5ex}
\begin{verbatim}
        head -5
\end{verbatim}
\end{frame}


\section{sed与awk}

\subsection{基础}

\begin{frame}[fragile]
\frametitle{awk和sed基础}

\emph{sed}和\emph{awk}是Unix环境中最强大的文本过滤工具, 他们有一些相似之处
\begin{itemize}
\item 激活语法相同:
    \begin{itemize}
    \item \verb~command 'script' filenames~
    \item command 为 \emph{sed}或\emph{awk}, script则是分别被\emph{sed}或\emph{awk}理解的命令清单
    \end{itemize}
\item 用户可为输入文件的每一行指定执行的指令
\item 为匹配模式使用正则表达式
\end{itemize}
\end{frame}

\begin{frame}
\frametitle{基本操作}

当\emph{sed}和\emph{awk}命令运行时, 执行如下操作
\begin{enumerate}
\item \label{sed.awk.op.1}从输入文件读入一行
\item 为该行做一个拷贝
\item 在该行上执行所给的脚本script
\item 处理下一行, 重复第\ref{sed.awk.op.1}步
\end{enumerate}
\end{frame}

\begin{frame}
\frametitle{脚本结构及执行}

\emph{sed}和\emph{awk}指定的脚本script包含一行或多行记录, 格式为:\\
/pattern/action\\
脚本执行:
\begin{enumerate}
\item \label{sed.awk.script.1}顺序搜索每个pattern, 直到发生一个匹配
\item 当发现匹配后, 为输入行执行相应的动作action
\item 当动作执行完毕, 到达下一个模式, 并重复第\ref{sed.awk.script.1}步
\item 当所有模式都试过后, 读取下一行
\end{enumerate}
\end{frame}

\subsection{sed}

\begin{frame}
\frametitle{sed}

\emph{sed}中常用的动作\\[1ex]

\begin{tabular}{p{5cm}l}\hline
动作 & 描述 \\ \hline
p & 打印该行 \\
d & 删除该行 \\
s/pattern1/pattern2/ & 用pattern2替换pattern1 \\
\{ COMMAND \} & 一组命令 \\
: LABEL & 定义标号 \\
b LABEL & 跳到标号 \\ \hline

\end{tabular}


\end{frame}


\begin{frame}
\frametitle{示例: 数据文件}

{cat fruit.txt}\\[1ex]
\begin{tabular}{p{2cm}l}
Fruit & Price/lbs \\
Banana & 0.89 \\
Paech & 0.79 \\
Kiwi & 1.50 \\
Pineapple & 1.29 \\
Apple & 0.99 \\
Mango & 2.20 \\
\end{tabular}

\end{frame}

\begin{frame}[fragile]
\frametitle{示例}

\begin{itemize}
\item 打印水果价格低于1美元的水果清单\\
\begin{Verbatim}
sed -n '/0\.[0-9][0-9]$/p' fruit.txt
\end{Verbatim}

\item 删除清单中关于芒果的信息\\
\begin{Verbatim}
sed '/^[Mm]ango/d' fruit.txt
\end{Verbatim}

\item 将清单中错误的Paech拼写纠正过来\\
\begin{Verbatim}
sed 's/Paech/Peach/' fruit.txt
\end{Verbatim}

\item 在清单中单价前面增加美元字符\\
\begin{Verbatim}
sed 's/[0-9][0-9]*\.[0-9][0-9]$/\$&/' fruit.txt
\end{Verbatim}
\end{itemize}

\end{frame}

\begin{frame}[fragile]
\frametitle{示例}
\begin{Verbatim}
find  /home/tolstoy -type d -print |
 sed 's;/home/tolstoy/;/home/lt/;' |
  sed 's/^/mkdir /'                |
   sh -x
\end{Verbatim}
\end{frame}

\subsection{awk}

\begin{frame}
\frametitle{awk}

\begin{itemize}
\item \emph{awk} 是一个完整的编程语言, 支持按模式搜索文件并有条件的改变文件
\item \emph{awk} 可以自动将输入行分隔为域(Field). 域是被一个或多个域分隔符分隔的字符串,
缺省域分隔符为TAB和空格
\item \emph{awk} 有三个主要版本
    \begin{itemize}
    \item 最初的\emph{awk}, 源自1978年的Unix V7
    \item 新版的\emph{nawk}, 1987年发布为SunOS 4.1的一部分
    \item POSIX/GNU版的\emph{gawk}
    \end{itemize}
\end{itemize}
\end{frame}

\begin{frame}
\frametitle{示例: 数据文件}

{cat fruit.txt}\\[1ex]
\begin{tabular}{p{2cm}p{3cm}l}
Fruit & Price/lbs & Quantity\\
Banana & \$0.89 & 100\\
Paech & \$0.79 & 65\\
Kiwi & \$1.50 & 22\\
Pineapple & \$1.29 & 35\\
Apple & \$0.99 & 78\\
Mango & \$2.20 & 46\\
\end{tabular}


\end{frame}

\begin{frame}[fragile]
\frametitle{示例--1}

\begin{itemize}
\item 打印水果名称及数量\\
  {\small \verb~awk '{ print $1, $3 }' fruit.txt~}\\
格式化输出:\\
{\small \verb~awk '{ printf "%-15s %s\n", $1, $3 }' fruit.txt~}

\item 在价格高于1美元的水果行后加上一个"*"\\
\begin{lstlisting}
#!/bin/sh
 awk '
    /\$[1-9][0-9]*\.[0-9][0-9]*/ { print $0, "*" }
    /\$0\.[0-9][0-9]*/ { print }
 ' fruit.txt
\end{lstlisting}
\end{itemize}
\end{frame}

\begin{frame}[fragile]
\frametitle{示例--2}

查找所有数量低于70的水果, 标记其为``REORDER''\\
\begin{lstlisting}
#!/bin/sh
 awk '
    $3 <= 70 { print $0, "REORDER" }
    $3 > 75 { print $0 }
 ' fruit.txt
\end{lstlisting}

\end{frame}


\begin{frame}[fragile]
\frametitle{示例--3}

查找单价高于1美元, 数量低于70的水果 \\
\begin{lstlisting}
#!/bin/sh
 awk '
    ($2 ~ /^\$[1-9][0-9]*\.[0-9][0-9]$/) && ($3 <= 70) {
        printf "%s\t%s\t%s\n", $0, "*", "REORDER"
    }
 ' fruit.txt
 \end{lstlisting}

\end{frame}

\section{parallel}

\begin{frame}[fragile]
    \frametitle{利用多核/多机并行执行脚本}
\begin{block}{GNU parallel}
一个 Linux 下的脚本工具,用来并行执行本地/远程机器上的作业.
\begin{itemize}
    \item 从标准输入读取, 每行输入启动一个命令或脚本执行. 
    \begin{itemize}
    \item 典型输入: 文件列表,  URL列表, 主机列表, 表格等
    \end{itemize}
    \item 缺省每核启动一个进程
    \item 可以对大文件自动分块处理, 然后合并结果
    \item 完全可替换 \verb~xargs, cat | bash~
\end{itemize}
\end{block}
\end{frame}


\begin{frame}[fragile]
    \frametitle{替换 xargs}
\begin{Verbatim}
find . -type d -print | xargs echo Dir:
find . -type d -print | parallel -X echo Dir:
find . -type d -print | xargs -I {} echo Dir: {}
find . -type d -print | parallel echo Dir: {}
\end{Verbatim}
\pause
\begin{Verbatim}
ls | xargs -t -L 4 echo
ls | parallel -t -L 4 echo
\end{Verbatim}
\pause
\begin{Verbatim}
find . -type d -name ".svn" -print | xargs rm -rf
find . -type d -name ".svn" -print | parallel rm -rf
\end{Verbatim}
\pause
\begin{Verbatim}
find . -type d -name ".svn" -print | parallel -m rm -rf
\end{Verbatim}
\end{frame}

\begin{frame}[fragile]
    \frametitle{从命令行读入}
\begin{itemize}
\item <1->压缩所有的html文件 \\
\begin{Verbatim}
parallel gzip ::: *.html
\end{Verbatim}
\item <2->将所有wav文件转化为mp3文件, 每个核启动一个进程 \\
\begin{Verbatim}
parallel lame {} -o {.}.mp3 ::: *.wav
\end{Verbatim}
\end{itemize}
\end{frame}

\begin{frame}[fragile]
    \frametitle{替换}
删除 pict0000.jpg到pict9999.jpg
\begin{itemize}
\item [] <1-> 逐个删除 \\
\begin{Verbatim}
seq -w 0 9999 | parallel rm pict{}.jpg 
\end{Verbatim}
\item [] <2-> 批量删除 \\
\begin{Verbatim}
seq -w 0 9999 | perl -pe 's/(.*)/pict$1.jpg/' |  \
     parallel -m rm
seq -w 0 9999 | parallel -X rm pict{}.jpg
\end{Verbatim}
\end{itemize}
\end{frame}

\begin{frame}[fragile]
    \frametitle{加速}
\noindent 比较
\begin{Verbatim}
seq -w 0 9999 | parallel touch pict{}.jpg
seq -w 0 9999 | parallel -X touch pict{}.jpg
\end{Verbatim}
\noindent 如果脚本无法接受多个参数, 可以启动多个parallel
\begin{Verbatim}
seq -w 0 999999 | parallel -j10 --pipe \
    parallel -j0 touch pict{}.jpg
\end{Verbatim}
\verb~-j0~如果启动512个进程, 则总共启动最多5120个进程
\end{frame}


\begin{frame}[fragile]
    \frametitle{结果收集}
GNU parallel缺省会将作业的输出收集后输出, 因此只有作业结束后结果才会输出. 如果希望作业在运行时也会输出, 可以使用\verb~-u~选项.\\

比较
\begin{Verbatim}
parallel traceroute ::: sf.net debian.org
\end{Verbatim}
和
\begin{Verbatim}
parallel -u traceroute ::: sf.net debian.org
\end{Verbatim}
\end{frame}

\begin{frame}[fragile]
    \frametitle{使用远程主机}
\alert{前提: 能够无密码ssh登陆远程主机, 可用ssh-copy-id设置}    
\begin{Verbatim}
seq 10 | parallel --sshlogin server.example.com echo
seq 10 | parallel --sshlogin 8/server.example.com echo
seq 10 | parallel --sshlogin user1@server.example.com \
      --sshlogin user1@server2.example.net echo
\end{Verbatim}
\end{frame}


\begin{frame}[fragile]
    \frametitle{传输文件}
\begin{Verbatim}
find logs/ -name '*.gz' | \
    parallel --sshlogin server1,server2,server3 \
        --transfer "zcat {} | bzip2 -9 >{.}.bz2"
\end{Verbatim}
\begin{Verbatim}
find logs/ -name '*.gz' | \
    parallel --sshlogin server.example.com \
        --transfer --return {.}.bz2 "zcat {} | \
            bzip2 -9 >{.}.bz2"
\end{Verbatim}
\begin{Verbatim}
find logs/ -name '*.gz' | \
    parallel -S server.example.com,: \
        --trc {.}.bz2 "zcat {} | bzip2 -9 >{.}.bz2"
\end{Verbatim}
\end{frame}

\section<article>{附加}

\begin{frame}<article>[fragile]
	\frametitle{expand/unexpand}
	expand将文件中的tab替换为空格, 输出到标准输出.

\begin{Verbatim}
expand -t 4 hello.c
\end{Verbatim}

\end{frame}

\begin{frame}<article>[fragile]
\frametitle{fmt: 重新格式化段落}
\begin{Verbatim}
$ fmt -s -w 10 << EOF
> one two three four five
> six
> seven
> eight
EOF
one two
three
four five
six
seven
eight
\end{Verbatim}
\end{frame}

\begin{frame}<article>[fragile]
    \frametitle{小技巧--1}

    \begin{tabular}{|l|p{6cm}|} \hline
脚本片段 & 效果 \\ \hline\hline
\verb~find /usr -print~ & 找出``/usr''下的所有文件  \\ \hline
\verb~seq 1 100~ & 打印1到100的整数 \\ \hline
\verb~|xargs -n 1 <cmd>~ & 以管道传过来的每项数据为参数, 重复运行cmd命令 \\ \hline
\verb~|xargs -n 1 echo~ & 将管道传过来的以空格分隔的数据逐行显示  \\ \hline
\verb~|xargs echo~ & 将管道传过来的所有数据合并成一行 \\ \hline
\verb~|cut -d: -f3 -~ & 将管道传过来的以:分隔的数据的第三列取出 \\ \hline
\end{tabular}

\end{frame}

\begin{frame}<article>[fragile]
    \frametitle{小技巧--2}

    \begin{tabular}{|p{5cm}|p{5cm}|} \hline
脚本片段 & 效果 \\ \hline\hline
\verb~|awk '{print $3}'~ &  将管道传过来的以空格分隔的数据的第三列取出  \\ \hline
\verb~|awk -F':' '{print $3}'~ &  将管道传过来的以:分隔的数据的第三列取出  \\ \hline
\verb~|sort | uniq ~ & 排序并去重 \\ \hline
\verb~|tr -d '\r' ~ & 删除 CR\\ \hline
\verb~|sed 's/^/#/'~ & 每行行首增加一个'\#'\\ \hline
\verb~|sed -n -e 2p~ & 打印第二行\\ \hline
\verb~|tail -n 2 -~ & 打印最后二行\\ \hline
\end{tabular}

\end{frame}

\end{document}
